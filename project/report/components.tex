\section{Components}
\subsection{System Call: sys\_pbarrier} % complete
The system call takes 3 arguments\\
1. operation\\
2. arg1\\
3. arg2\\\\
It then obtains the task\_struct pointer of the calling process using macro `current'. A hook function is called and this task\_struct pointer along with the 3 arguments given above are passed into it.

\subsection{LKM: barrier\_sys.c} % complete
The loadable kernel module implements the hook function \texttt{pbarrier}. It also initializes a linked list of barrier groups and a sysfs interface for providing current stats. The parts of the lkm are given below.
\subsubsection{Linked List}
A linked list of barrier groups is implemented in `\texttt{grptskds.c}'. A barrier-element in the linked list has 3 members\\
1. gid\ \ \ \ : Group ID\\
2. nproc: No. of processes for the barrier group\\
3. tsks\ \ \ : A linked list of \textbf{task\_struct pointers} --- tasks blocked on the barrier\\

Various functionalities possible in the linked list are:
\begin{enumerate}
\item \texttt{createGroup(gid, nproc)}: creates a new group with given gid and nproc, and adds it to the list. It then returns the address of that group element. Returns NULL if that gid is already taken.
\item \texttt{getGroup(gid)}: returns the linked list element (group) with the given gid. Returns NULL if not found.
\item \texttt{insertTask(gid, tsk)}: inserts the task\_struct `tsk' into the group-node with id `gid'. As the task is inserted, the counter `nproc' is decremented. Returns the following values\\
\begin{enumerate}
	\item 0: task inserted and nproc decremented successfully.\\
	\item 1: nproc reached zero, denoting it was the last process for the group. All processes in the group must be freed.\\
	\item 2: group id not present.\\
	\item 3: task already present in the group.\\
	\item 4: invalid (negative) gid.\\
\end{enumerate}
\item \texttt{printGroups(gid, buf)}: if gid is -1, state of all groups is entered into the buffer `buf'. Otherwise, state of that group with given gid is copied into `buf'.
\item \texttt{freeGroup(gid)}: Every process blocked in the barrier group `gid' is allowed to continue and the `tsks' list is freed. The group element in the linked list is also freed. The given gid is free to be used again.
\item \texttt{freeLists()}: All the groups' task\_struct list tsks is freed, and the linked list of group-elements are also safely freed.
\end{enumerate}

\subsubsection{pbarrier function}
The function takes 4 arguments: the \texttt{task\_struct} of the process to be blocked, an \texttt{operation code}, and two arguments \texttt{arg1} and \texttt{arg2}. A spin lock is used for mutual exclusion. An appropriate action is taken according to the operation code. \\
If the operation code is 0 (BARRIER\_INIT): \\
arg1 is considered as gid and arg2 as nproc. \texttt{createGroup(arg1,arg2)} is called and a new group is created with given gid and nproc.\\
If the operation code is 1 (BARRIER\_BLOCK):\\
arg1 is taken as gid and arg2 is ignored. \texttt{insertTask(task, arg1)} is called and the task is added to the task-list of group with given `gid'. If the function returns 0, the process is blocked by sending \textbf{SIGSTOP} . If the function returns 1, all the tasks blocked in that gid is unblocked and the lists freed by calling \texttt{freeGroup(arg1)}.

\subsubsection{sysfs interface}
Two sysfs files have been created in \texttt{/sys/barrier/}: \texttt{gid} and \texttt{stats}. gid is initialized to -1.\\
Inorder to obtain stats of all the existing groups, set gid to -1 (by writing into \texttt{gid} file) and read \texttt{stats} file.
\begin{verbatim}
echo "-1" > /sys/barrier/gid
cat /sys/barrier/stats
\end{verbatim}
To obtain stats of group with id `gid', write that into \texttt{gid} file and read the \texttt{stats} file.
\begin{verbatim}
echo {gid} > /sys/barrier/gid
cat /sys/barrier/stats
\end{verbatim}

\subsection{The Header File: pbarrier.h} % complete
A header file \textbf{pbarrier.h} was created which has wrapper functions \texttt{initGroup(gid, nproc)} and \texttt{block(gid)}, which internally calls the system call `\texttt{pbarrier}'.
\subsubsection{initGroup(gid, nproc)}
Initiates a new barrier group with group-id \texttt{gid} and number of processes \texttt{nproc}. A call with an existing group id will be neglected. When all the `nproc' processes in the group reaches and blocks at the barrier, the processes will be allowed to continue, the group will be cleared, and the group id `gid' will be free for a new group.
\subsubsection{block(gid)}
Blocks the caller process at the barrier of group `gid'. If it was the last process out of the `nproc' processes of that barrier, all the processes blocked at `gid' barrier group will be released and they continue execution.
